\documentclass{scrartcl}
\usepackage[utf8]{inputenc}
\usepackage[nottoc]{tocbibind}
\usepackage{pdflscape}
\usepackage[figurename=Fig.]{caption}
\usepackage{graphicx}
\usepackage{indentfirst}
\usepackage{enumitem}
\setkomafont{disposition}{\normalfont\bfseries}
\usepackage{titlesec}

\setcounter{secnumdepth}{4}
\setcounter{tocdepth}{6}

\titleformat{\paragraph}
{\normalfont\normalsize\bfseries}{\theparagraph}{1em}{}
\titlespacing*{\paragraph}
{0pt}{3.25ex plus 1ex minus .2ex}{1.5ex plus .2ex}
%opening
\title
{
	\Huge UNIVERSIDAD NACIONAL DE CÓRDOBA \\
	\vspace{1cm} \LARGE Facultad de Ciencias Exactas, Físicas y Naturales
}       

\subtitle
{
	\vspace{1cm} \large   Proyecto Integrador \\
	Ingeniería en Computación \\~\\
	Predicción automática del clima basada en redes neuronales convolucionales
	\vspace{1cm}
}

\author
{
	\textbf{Autores}\\
	\textbf{Ortmann, Nestor Javier - Trejo, Bruno Guillermo}\\
}

\setcounter{page}{0}

\renewcommand*\contentsname{Indice}

\date{Córdoba, Argentina \\ 2019}

\begin{document}
	\maketitle
	\thispagestyle{empty}
	 
	\begin{figure}[t!]
		\begin{center}
			\includegraphics[width=8cm]{images/logoUNC.jpg}
		\end{center}
	\end{figure}
	
	\begin{center}
	\large\textbf{Director\\PhD. Micolini, Orlando\\Codirector\\Ing. Luis O. Ventre}
	\end{center}
	\cleardoublepage
	
	\tableofcontents% indice de contenidos
	\cleardoublepage
	
	\section{Introducción}
		\begin{subsection}{Estado del Arte}
			Aca iria estado del arte.     
		\end{subsection}
		
		\begin{subsection}{Motivación}

			\medskip El Machine Learning (ML o Aprendizaje Automático) es una de las tecnologías más trascendentes 
			que han resurgido en los últimos 10 años impulsada principalmente por el crecimiento exponencial
			de los datos producidos por las personas que hacen uso de muchas tecnologías relacionadas a internet.\cite{ortega}
			Hoy en día disponemos de suficiente poder de cómputo a un precio accesible para extraer patrones de 
			conjuntos masivos de datos (Big Data) y obtener valiosa información de los mismos, lo cual 
			nos permite solucionar muchos problemas de ingeniería utilizando en ellos estos algoritmos de aprendizaje.
			
			\medskip A lo largo del ejercicio de su actividad, un ingeniero muchas veces se encuentra en la situación de tener
			una gran variedad de herramientas, 
			métodos o algoritmos que a priori podrían solucionar el problema que se le presenta. 
			Sin embargo, cada una de estas herramientas tienen pros y contras 
			en su desarrollo que llevan al profesional a tener que seleccionar la que a su criterio sea la mejor según
			el objetivo que busque, los datos y recursos que posea.
			Este escenario nos motiva y genera una necesidad de encontrar un panorama claro y a la vez amplio que le 
			permita al lector, y a nosotros mismos, identificar las características que tiene el proyecto y de esta 
			forma ayudarnos en el proceso de selección de un algoritmo adecuado.
		\end{subsection}
		
		\cleardoublepage

		\begin{subsection}{Objetivos}
			Aca irian los objetivos.
		\end{subsection}

		\begin{subsection}{Requerimientos}
			\begin{subsubsection}{Requerimientos Funcionales}
				Requerimientos funcionales
				\begin{table}[h]
					\begin{tabular}{|l|l|}
					\hline
					ID  & Descripción \\ \hline \hline
					RF1 & Debe leer los archivos wrf (modelo meteorológico) y obtener una matriz numpy con \\& las variables 
						  de entrada.\\ \hline
					RF1 & Debe leer los archivos de hoja de calculo de los datos medidos y obtener una matriz \\& numpy de 
						  etiquetas (variables de salida).\\ \hline
					RF2 & Debe construir una red neuronal utlizando las matrices de variables de entrada y las \\& etiquetas 
						  para su entrenamiento\\ \hline
					RF3 & proximamente...\\ \hline
					\end{tabular}
					\end{table}
				\begin{paragraph}{Prueba de Requerimientos Funcionales}
					Aca irian las pruebas.
				\end{paragraph}				
			\end{subsubsection}

			\begin{subsubsection}{Requerimientos No Funcionales}
				Requerimientos no funcionales
				\begin{table}[h]
					\begin{tabular}{|l|l|}
					\hline
					ID   & Descripción \\ \hline \hline
					RNF1 & El tiempo que tarda en crear la matriz de valores de entrada debe ser menor a 1 \\& minuto.\\ \hline
					RNF2 & proximamente...\\ \hline
					\end{tabular}
				\end{table}
			\end{subsubsection}
		\end{subsection}
		\cleardoublepage
		\begin{subsection}{Análisis de Riesgos}
			Aca iria la explicacion teorica del metodo.
			\begin{subsubsection}{Listado de riesgos}
				\begin{table}[!h]
					\begin{tabular}{|l|l|}
					\hline
					ID & Description\\  \hline \hline
					R1 & encontrar el modelo (CNN) adecuado para encarar el problema\\  \hline
					R2 & encontrar la herramienta (keras) para usar dicho modelo\\  \hline
					R3 & obtener los datos (tanto de entrada como de salida)\\  \hline
					R4 & requerimientos de hardware (computo, espacio en disco, memoria)\\  \hline
					R5 & poder trabajar remotamente desde el laboratorio\\  \hline
					R6 & proximamente....\\  \hline
					\end{tabular}
				\end{table}
			\end{subsubsection}
			\begin{subsubsection}{Estimación de probabilidad}
				Estimar los riesgos.
				\begin{table}[!h]
					\begin{tabular}{|l|l|l|}
					\hline
					ID & Riesgo & Probabilidad \%\\  \hline \hline
					R1 & encontrar el modelo (CNN) adecuado para encarar el problema & 30\\  \hline
					R2 & encontrar la herramienta (keras) para usar dicho modelo& 5\\  \hline
					R3 & obtener los datos (tanto de entrada como de salida)& 70\\  \hline
					R4 & requerimientos de hardware (computo, espacio en disco, memoria)& 45\\  \hline
					R5 & poder trabajar remotamente desde el laboratorio& 30\\  \hline
					R6 & proximamente...& NaN\\  \hline
					\end{tabular}
				\end{table}
			\end{subsubsection}
			\begin{subsubsection}{Estimación de impacto}
				Estimar el impacto.
			\end{subsubsection}
			\begin{subsubsection}{Exposición al riesgo}
				Exposicion.
			\end{subsubsection}
		\end{subsection}

		\begin{subsection}{Arquitectura preliminar de alto nivel}
			Dibujo de arquitectura.
		\end{subsection}
	
	\section{Marco Teórico}
		Teoría.
	
		\begin{subsection}{Subseccion}
			Aca iria una subseccion
		\end{subsection}
	
	\section{Iteración 0}

	\section{Iteración N: algo}
	
	\section{Conclusión}
		En este trabajo se introdujeron los conceptos fundamentales relacionados al aprendizaje automático
		de manera que el lector se familiarice con los mismos y pueda tener el marco teórico necesario 
		para solucionar un problema utilizando las librerías existentes. Se realizó un análisis del 
		estado del arte de las herramientas en el área de Machine Learning, investigando las distintas 
		librerías, aplicaciones y alternativas más usadas en estos últimos años.
		
		\medskip De esta manera, se hace evidente lo útil que es tener un punto de partida y una guía que permita bajar
		a tierra los conocimientos aprendidos. La gráfica desarrollada posibilita obtener esto, brindando una
		clasificación de los distintos algoritmos a utilizar y un método de selección rápido y preciso.
		
		\medskip Con este panorama claro y general y con la ayuda visual para seleccionar la herramienta
		adecuada en base a los requerimientos del proyecto, se puede avanzar en la resolución de cualquier
		problema específico donde se quiera implementar Machine Learning. De todos modos, se pretende seguir
		mejorando y actualizando la gráfica a medida que surjan nuevas herramientas, para mantenerse a la vanguardia de la 
		tecnología.

	\cleardoublepage

	\bibliographystyle{unsrt}
	\bibliography{bibliografia}	

\end{document}
